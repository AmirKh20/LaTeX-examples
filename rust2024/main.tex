\documentclass[10pt]{beamer}

\usetheme{AnnArbor}
\usecolortheme{dracula}


\title{The Rust Programming Language}
\author{Amir~Khayatzadeh}
\institute[IAUM]{Department of Computer Engineering\\ Islamic Azad University, Mashhad Branch}
\date{Spring 2024}

\usepackage{tikz}
\titlegraphic{
    \begin{tikzpicture}[overlay, remember picture]
    \node[right=0.05cm] at (current page.149){
        \includegraphics[scale=.1]{./images/azad-logo.png}
    };
    \node[right=1.5cm] at (current page.151){
        {\scriptsize Islamic Azad University, Mashhad Branch}
    };
    \end{tikzpicture}
}

\AtBeginSection[]
{
  %\setbeamerfont{frametitle}{size=\tiny}
    \begin{frame}[allowframebreaks]
    %\fontsize{4pt}{7}\selectfont
    %\vspace*{-5mm}
    \frametitle{Table of Contents}
    \tableofcontents[currentsection]
  \end{frame}
}

\begin{document}

\begin{frame}[plain]
    \titlepage
\end{frame}

%\section{Introduction}
%\begin{frame}{Why Rust?\cite{rust-website}}
%
%    \begin{columns}
%
%        \begin{column}{0.5\textwidth}
%            \begin{itemize}
%                \item Performance
%                    \begin{itemize}
%                        \item Fast and Memory Efficient
%                        \item No Garbage Collector
%                        \item Embedded Devices
%                        \item In Linux Kernel
%                    \end{itemize}
%                \item Reliability
%                    \begin{itemize}
%                        \item Rich Type System
%                        \item Ownership Model
%                    \end{itemize}
%                \item Productivity
%                    \begin{itemize}
%                        \item Friendly Compiler
%                        \item Package Manager
%                        \item Build Tool
%                        \item Auto Formatter
%                    \end{itemize}
%            \end{itemize}
%        \end{column}
%
%        \begin{column}{0.5\textwidth}
%            \begin{figure}
%                \centering
%                \includegraphics[scale=.1]{images/rust-logo.png}
%                \caption{Rust Icon \cite{pngwing}}
%            \end{figure}
%        \end{column}
%
%    \end{columns}
%
%\end{frame}

%\defverbatim[colored]\makeset{
%    \begin{lstlisting}[language=Rust, keywordstyle=\color{blue}]
%        fn main() {
%            let x = 5;
%            println!("The value of x is: {x}");
%            x = 6;
%            println!("The value of x is: {x}");
%        }
%    \end{lstlisting}
%}
\section{Variables and Mutability}
\subsection{Immutable Variables}
\begin{frame}
    By default, all variables in Rust are immutable. When a variable is immutable, once a value is bound to a name, you can’t change that value.\cite{rust-book}

    \begin{figure}[htpb]
        \centering
        \includegraphics[width=0.45\textwidth]{./images/immutable-variable.png}
        \vspace*{-3mm}
        \caption{Filename: src/main.rs\cite{rust-book}}
    \end{figure}

    \vspace*{-5mm}

    \begin{figure}[htpb]
        \centering
        \includegraphics[width=0.85\textwidth]{./images/immutable-variable-output.png}
        \vspace*{-2mm}
        \caption{Error Message On Compilation\cite{rust-book}}
    \end{figure}

\end{frame}

\subsection{Mutable Variables}
\begin{frame}
    Put "mut" before the identifier to make the variable mutable.
    \begin{figure}[htpb]
        \centering
        \includegraphics[width=0.9\textwidth]{./images/mutable-variable.png}
        \caption{Filename: src/main.rs\cite{rust-book}}
    \end{figure}

    \begin{figure}[htpb]
        \centering
        \includegraphics[width=\textwidth]{./images/mutable-variable-output.png}
        \caption{Output of The Code\cite{rust-book}}
    \end{figure}

\end{frame}

\subsection{Constants}
\begin{frame}
    \begin{figure}[htpb]
        \centering
        \includegraphics[width=0.8\textwidth]{./images/constant.png}
        \caption{An example of a constant declaration.\cite{rust-book}}
    \end{figure}

    Differences between immutable variables\cite{rust-book}:
    \begin{enumerate}
        \item The constant's type must be annotated.
        \item Constants are compile-time. Immutable variables are runtime.
        \item Constants are set only to a constant expression, not the result of a value that could only be computed at runtime.
        \item Immutable variables cannot be global.
    \end{enumerate}

    \begin{figure}[htpb]
        \centering
        \includegraphics[width=\textwidth]{./images/global-immutable.png}
        \caption{Compilation Error For Declaring An Immutable Variable Globally.}
    \end{figure}
\end{frame}


\section{Data Types}
\begin{frame}{Rust, A Statically Typed Language}
    Compiler usually can infer the type based on the value, but in cases where multiple types are possible, we need to annotate the type:

    \begin{figure}[htpb]
        \centering
        \includegraphics[width=0.8\textwidth]{./images/convert-type.png}
        \vspace*{-2mm}
        \caption{Convert a string to a numeric type.\cite{rust-book}}
    \end{figure}

    \vspace*{-5mm}
    \begin{figure}[htpb]
        \centering
        \includegraphics[width=0.8\textwidth]{./images/compiler_error_convert_type.png}
        \caption{If the type is not specified, we get compiler error.\cite{rust-book}}
    \end{figure}
\end{frame}

\subsection{Scalar Types}
\begin{frame}{Integer Types}
    \vspace*{-5mm}
    \begin{figure}[htpb]
        \centering
        \includegraphics[width=0.6\textwidth]{./images/integer-types-table.png}
        \caption{Integer Types in Rust\cite{rust-book}}
    \end{figure}
    the isize and usize types depend on the architecture of the computer your program is running on, which is denoted in the table as “arch”: 64 bits if you’re on a 64-bit architecture and 32 bits if you’re on a 32-bit architecture.\cite{rust-book}
\end{frame}

\begin{frame}{Floating-Point Types}
    \begin{figure}[htpb]
        \centering
        \includegraphics[width=0.8\textwidth]{./images/floating-point.png}
        \caption{An example of floating-point variables in rust.\cite{rust-book}}
    \end{figure}
    The default type is f64 because on modern CPUs, it’s roughly the same speed as f32 but is capable of more precision. All floating-point types are signed.\cite{rust-book}


\end{frame}

\begin{frame}{The Boolean and Character Types}
    \begin{figure}[htpb]
        \centering
        \includegraphics[width=0.9\textwidth]{./images/boolean.png}
        \caption{An example of declaring boolean values.\cite{rust-book}}
    \end{figure}

    \vspace*{-5mm}
    \begin{figure}[htpb]
        \centering
        \includegraphics[width=0.9\textwidth]{./images/character-type.png}
        \caption{An example declaring char values.\cite{rust-book}}
    \end{figure}

    \vspace*{-5mm}
    \begin{itemize}
        \item Declared using single quotations.
        \item 4 bytes in size.
        \item Supports Unicode values.
    \end{itemize}
\end{frame}

\subsection{Compound Types}
\begin{frame}{The Tuple Type}
    \begin{itemize}
        \item Groups of values with different types.
        \item Tuples have a fixed length.
    \end{itemize}

    \begin{figure}[htpb]
        \centering
        \includegraphics[width=0.8\textwidth]{./images/tuple.png}
        \vspace*{-2mm}
        \caption{An example with optional type annotations.\cite{rust-book}}
    \end{figure}

    \vspace*{-5mm}
    \begin{figure}[htpb]
        \centering
        \includegraphics[width=0.8\textwidth]{./images/tuple-example.png}
        \vspace*{-2mm}
        \caption{An example of destructuring a tuple.\cite{rust-book}}
    \end{figure}
\end{frame}

\begin{frame}{The Array Type}
    \begin{itemize}
        \item Groups of values with the same type.
        \item Arrays have a fixed length.
        \item Allocated on the stack rather than the heap.
    \end{itemize}

    \begin{figure}[htpb]
        \centering
        \includegraphics[width=0.8\textwidth]{./images/array-declartion1.png}
        \vspace*{-2mm}
        \caption{Declaring an array with type annotation and length.\cite{rust-book}}
    \end{figure}

    \vspace*{-5mm}
    \begin{figure}[htpb]
        \centering
        \includegraphics[width=0.6\textwidth]{./images/accessing-array.png}
        \vspace*{-2mm}
        \caption{Accessing array elements.\cite{rust-book}}
    \end{figure}
\end{frame}

\begin{frame}{An Example of Rust's Memory Safety}
    \begin{figure}[htpb]
        \centering
        \includegraphics[width=0.8\textwidth]{./images/array-index-out-of-range.png}
        \caption{An example of accessing array elements from input.\cite{rust-book}}
    \end{figure}
\end{frame}

\begin{frame}{Panicking}
    \begin{figure}[htpb]
        \centering
        \includegraphics[width=\textwidth]{./images/rust-panicking-array.png}
        \caption{Runtime Error (Panicking) if 10 is given as the index.\cite{rust-book}}
    \end{figure}

    This is an example of Rust’s memory safety principles in action. In many low-level languages, this kind of check is not done, and when you provide an incorrect index, invalid memory can be accessed.\cite{rust-book}

\end{frame}

\section{Functions}
\begin{frame}{Definition And Parameters Declaration}
    \begin{figure}[htpb]
        \centering
        \includegraphics[width=\textwidth]{./images/function-declaration-parameters.png}
        \caption{An example of defining functions with parameters.\cite{rust-book}}
    \end{figure}
\end{frame}

\subsection{Statements and Expressions}
\begin{frame}{Rust: An Expression-Based Language}{Statements}
    \begin{itemize}
        \item Statements are instructions that perform some action and do not return a value.
    \end{itemize}

    Creating a variable and assigning a value to it with the let keyword is a statement.

    Statements do not return values. Therefore, you can’t assign a let statement to another variable.
    \begin{figure}[htpb]
        \centering
        \includegraphics[width=0.5\textwidth]{./images/statement-example1.png}
        \caption{Code Using The let Statement In The Wrong Way.\cite{rust-book}}
    \end{figure}

    \vspace*{-5mm}
    \begin{itemize}
        \item Different from C and other languages, where the assignment returns the value of the assignment.
        \item In other languages, you can write x = y = 6; not in Rust.
    \end{itemize}
\end{frame}

\begin{frame}{Rust: An Expression-Based Language}{Expressions}
    \begin{itemize}
        \item Expressions evaluate to a resultant value.\cite{rust-book}
    \end{itemize}
    Consider a math operation, such as 5 + 6, which is an expression that evaluates to the value 11.\cite{rust-book}

    \begin{itemize}
        \item The 6 in the statement let y = 6; is an expression that evaluates to the value 6.
        \item Calling a function/macro is an expression.
        \item Scope blocks are expressions.
    \end{itemize}

    \begin{columns}
        \begin{column}{0.5\textwidth}

            \begin{figure}[htpb]
                \centering
                \includegraphics[width=\textwidth]{./images/expression-example1.png}
                \caption{An example showcasing expressions.\cite{rust-book}}
            \end{figure}

        \end{column}

        \begin{column}{0.5\textwidth}

            \begin{itemize}
                \item The x + 1 line doesn’t have a semicolon at the end.
                \item Expressions do not include ending semicolons.
                \item Adding semicolons turn it to a statement, which do not return a value.
            \end{itemize}

        \end{column}
    \end{columns}

\end{frame}

\subsection{Functions with Return Values}
\begin{frame}{Examples of Functions that Return Values}
    \begin{columns}
        \begin{column}{0.5\textwidth}
            \vspace*{-3mm}
            \begin{figure}[htpb]
                \centering
                \includegraphics[width=\textwidth]{./images/function-return.png}
                \caption{A function that returns a value.\cite{rust-book}}
            \end{figure}

        \end{column}
        \begin{column}{0.5\textwidth}
            \begin{figure}[htpb]
                \centering
                \includegraphics[width=\textwidth]{./images/function-return2.png}
                \caption{A function that add 1 and returns the result.\cite{rust-book}}
            \end{figure}

        \end{column}
    \end{columns}

\end{frame}

\begin{frame}{A Common Mistake When Returning Values}
    \begin{figure}[htpb]
        \centering
        \includegraphics[width=0.5\textwidth]{./images/function-return-mistake.png}
        \vspace*{-2mm}
        \caption{Adding Semicolon to The Function's Last Expression.\cite{rust-book}}
    \end{figure}

    \vspace*{-3mm}
    \begin{figure}[htpb]
        \centering
        \includegraphics[width=0.8\textwidth]{./images/function-return-mistake-error.png}
    \end{figure}
\end{frame}

\section{Comments}
\begin{frame}{How to Write Comments?}
    \begin{figure}[htpb]
        \centering
        \includegraphics[width=\textwidth]{./images/comments.png}
        \caption{A simple comment\cite{rust-book}}
    \end{figure}

    \begin{figure}[htpb]
        \centering
        \includegraphics[width=\textwidth]{./images/comments2.png}
        \caption{Multi-Line Comment\cite{rust-book}}
    \end{figure}
\end{frame}

\section{Control Flows}
\subsection{If}
\begin{frame}{If Expression}
    \begin{figure}[htpb]
        \centering
        \includegraphics[width=0.6\textwidth]{./images/if.png}
        \vspace*{-2mm}
        \caption{An example using the if expression\cite{rust-book}}
    \end{figure}

    \vspace*{-5mm}
    \begin{figure}[htpb]
        \centering
        \includegraphics[width=0.6\textwidth]{./images/if2.png}
        \vspace*{-2mm}
        \caption{An erroneous example using the if expression\cite{rust-book}}
    \end{figure}
\end{frame}

\begin{frame}{Using if in a let Statement}
    \begin{figure}[htpb]
        \centering
        \includegraphics[width=0.8\textwidth]{./images/if3.png}
        \caption{Assigning the result of an if expression to a variable\cite{rust-book}}
    \end{figure}

    \begin{figure}[htpb]
        \centering
        \includegraphics[width=0.8\textwidth]{./images/if4.png}
        \caption{An erroneous example using the if let statement\cite{rust-book}}
    \end{figure}
\end{frame}

\subsection{Loops}
\begin{frame}{Infinite Loop}
    \begin{figure}[htpb]
        \centering
        \includegraphics[width=\textwidth]{./images/loop1.png}
        \caption{An example using the loop keyword\cite{rust-book}}
    \end{figure}
\end{frame}

\begin{frame}{Returning Values from Loops}
    \begin{figure}[htpb]
        \centering
        \includegraphics[width=0.8\textwidth]{./images/loop2.png}
        \caption{An example that returns a value in a loop\cite{rust-book}}
    \end{figure}
\end{frame}

\begin{frame}{Loop Labels}
    \begin{figure}[htpb]
        \centering
        \includegraphics[width=0.7\textwidth]{./images/loop3.png}
        \vspace*{-2mm}
        \caption{An example using loop labels\cite{rust-book}}
    \end{figure}
\end{frame}

\begin{frame}{Conditional Loops with while}
    \begin{figure}[htpb]
        \centering
        \includegraphics[width=0.7\textwidth]{./images/loop4.png}
        \caption{Using a while loop to run code while a condition holds true\cite{rust-book}}
    \end{figure}
\end{frame}

\begin{frame}{Looping Through a Collection with for}
    \begin{figure}[htpb]
        \centering
        \includegraphics[width=0.8\textwidth]{./images/loop5.png}
        \caption{Looping through each element of a collection using a for loop\cite{rust-book}}
    \end{figure}
\end{frame}

\section{Ownership: The Rust's Most Unique Feature}
\subsection{Definition}
\begin{frame}{Understanding}
    Why?
    \begin{itemize}
        \item Making memory safety guarantees without needing a garbage collector.\cite{rust-book}
        \item None of the features of ownership will slow down your program while it’s running.\cite{rust-book}
    \end{itemize}

    Topics to talk about:
    \begin{itemize}
        \item Borrowing.
        \item Slices.
        \item How Rust lays data out in memory.
    \end{itemize}

    What?
    \begin{itemize}
        \item A set of rules governing how a Rust program manages memory.\cite{rust-book}
        \item If any of the rules are violated, the program won’t compile.\cite{rust-book}
        \item The main purpose of ownership is to manage heap data.\cite{rust-book}
    \end{itemize}
\end{frame}

\begin{frame}
    Ownership rules:
    \begin{itemize}
        \item Each value in Rust has an owner.
        \item There can only be one owner at a time.
        \item When the owner goes out of scope, the value will be dropped.
    \end{itemize}

    The memory is automatically returned once the variable that owns it goes out of scope.\cite{rust-book}

    \begin{figure}[htpb]
        \centering
        \includegraphics[width=\textwidth]{./images/ownership1.png}
        \caption{An example of declaring a string in heap.\cite{rust-book}}
    \end{figure}

    After going out of scope, Rust calls the \textit{drop} function.
\end{frame}

\subsection{Variables and Data Interacting with Move}
\begin{frame}{Assignment of Variables}
    \begin{figure}[htpb]
        \centering
        \includegraphics[width=0.5\textwidth]{./images/ownership2.png}
        \caption{Assigning the integer value of variable x to y.\cite{rust-book}}
    \end{figure}
    Because integers are simple values with a known, fixed size, and these two 5 values are pushed onto the stack.\cite{rust-book}

    \begin{figure}[htpb]
        \centering
        \includegraphics[width=0.8\textwidth]{./images/ownership3.png}
        \caption{Doing the same with the String type.\cite{rust-book}}
    \end{figure}

    \begin{itemize}
        \item These two are not the same.
        \item The second line does not make a copy of the value in s1 and does not bind it to s2.
    \end{itemize}
\end{frame}

\begin{frame}{What Happens to String?}
    \vspace*{-3.5mm}
    \begin{columns}
        \begin{column}{0.5\textwidth}
            \begin{figure}[htpb]
                \centering
                \includegraphics[width=\textwidth]{./images/ownership4.png}
                \caption{Representation in memory of a String holding the value "hello" bound to s1\cite{rust-book}}
            \end{figure}

        \end{column}
        \begin{column}{0.5\textwidth}
            \begin{figure}[htpb]
                \centering
                \includegraphics[width=\textwidth]{./images/ownership5.png}
                \caption{Representation in memory of the variable s2 that has a copy of the pointer, length, and capacity of s1\cite{rust-book}}
            \end{figure}
        \end{column}
    \end{columns}
\end{frame}

\begin{frame}{Double Free Error}
    Both s1 and s2 point to the same memory location, leading to two times calling the drop function when they go out of scope.

    To solve this problem and ensure memory safety:
    \begin{itemize}
        \item After the line let s2 = s1;, Rust considers s1 as no longer valid.
        \item Therefore, Rust doesn’t need to free anything when s1 goes out of scope.
    \end{itemize}

    \begin{columns}
        \begin{column}{0.4\textwidth}
            \begin{figure}[htpb]
                \centering
                \includegraphics[width=\textwidth]{./images/ownership6.png}
                \caption{Erroneous code that tries to copy s1 to s2\cite{rust-book}}
            \end{figure}
        \end{column}
        \begin{column}{0.6\textwidth}
            \begin{figure}[htpb]
                \centering
                \includegraphics[width=\textwidth]{./images/ownership6-error.png}
                \caption{Rust prevents using an invalidated reference\cite{rust-book}}
            \end{figure}
        \end{column}
    \end{columns}

\end{frame}

\begin{frame}{Shallow Copy vs. Move}
    \begin{columns}
        \begin{column}{0.5\textwidth}
            \begin{figure}[htpb]
                \centering
                \includegraphics[width=\textwidth]{./images/ownership7.png}
                \caption{Representation in memory after s1 has been invalidated\cite{rust-book}}
            \end{figure}
        \end{column}
        \begin{column}{0.5\textwidth}
            \begin{itemize}
                \item Shallow Copy and Deep Copy
                \item In Rust it is a \textit{move} because the first variable gets invalidated.\cite{rust-book}
                \item We say that s1 was \textit{moved} into s2.\cite{rust-book}
                \item Rust will never automatically create “deep” copies of your data.\cite{rust-book}
                \item Therefore, any automatic copying can be assumed to be inexpensive in terms of runtime performance.\cite{rust-book}
            \end{itemize}
        \end{column}
    \end{columns}

\end{frame}

\subsection{Variables and Data Interacting with Clone}
\begin{frame}{Using Clone to Copy Deeply}
    If copying the heap data is desired, we can use a method called clone.
    \begin{figure}[htpb]
        \centering
        \includegraphics[width=\textwidth]{./images/ownership8.png}
        \caption{Using The Clone Method\cite{rust-book}}
    \end{figure}

    \begin{itemize}
        \item When you see a call to clone, you know that some arbitrary code is being executed and that code may be expensive.\cite{rust-book}
    \end{itemize}
\end{frame}

\subsection{Stack-Only Data: Copy}
\begin{frame}{Types with The Copy Trait}
    \begin{columns}
        \begin{column}{0.4\textwidth}
            Why in this code, x is still valid and wasn't moved into y?
        \end{column}
        \begin{column}{0.6\textwidth}
            \begin{figure}[htpb]
                \centering
                \includegraphics[width=\textwidth]{./images/ownership9.png}
                \caption{An example of copying variables with simple types\cite{rust-book}}
            \end{figure}
        \end{column}
    \end{columns}

    \begin{itemize}
        \item They are stored entirely on the stack.
        \item Copying of these variables is quick and inexpensive.
    \end{itemize}

    For variables that are stored in the stack, Rust has a special annotation called the Copy trait. If a type implements this trait, variables that use it do not move.

    \begin{itemize}
        \item All the integer types, such as u32.
        \item The Boolean type, bool, with values true and false.
        \item Tuples, if they only contain types that also implement Copy. For example, (i32, i32) implements Copy, but (i32, String) does not.
    \end{itemize}

\end{frame}

\subsection{Ownership and Functions}
\begin{frame}{Passing is Like Assigning}
    Passing a variable to a function will move or copy, just as assignment does.\cite{rust-book}

    \begin{figure}[htpb]
        \centering
        \includegraphics[width=0.85\textwidth]{./images/ownership-function.png}
        \vspace*{-2mm}
        \caption{Functions with ownership and scope annotated\cite{rust-book}}
    \end{figure}
\end{frame}

\begin{frame}{Return Values and Scope}
    Returning values can also transfer ownership.\cite{rust-book}
    \begin{figure}[htpb]
        \centering
        \includegraphics[width=0.7\textwidth]{./images/ownership-function2.png}
        \vspace*{-2mm}
        \caption{Transferring ownership of return values\cite{rust-book}}
    \end{figure}
\end{frame}

\begin{frame}
    It’s quite annoying that anything we pass in also needs to be passed back if we want to use it again.\cite{rust-book}

    Rust allows returning multiple values as a tuple:
    \begin{figure}[htpb]
        \centering
        \includegraphics[width=\textwidth]{./images/ownership-function3.png}
        \caption{Returning ownership of parameters}
    \end{figure}

    But this is too much work, so a feature is needed to address this problem.
\end{frame}

\section{References and Borrowing}
\begin{frame}{References}
    \begin{itemize}
        \item Like pointers.
        \item That data is owned by some other variable.
        \item Unlike a pointer, a reference is guaranteed to point to a valid value.
    \end{itemize}

    \begin{figure}[htpb]
        \centering
        \includegraphics[width=0.8\textwidth]{./images/references1.png}
        \caption{Using a reference to pass the value without loosing ownership\cite{rust-book}}
    \end{figure}
\end{frame}

\begin{frame}{References}{In The Memory}
    \begin{figure}[htpb]
        \centering
        \includegraphics[width=\textwidth]{./images/references2.png}
        \caption{A diagram of \&String s pointing at String s1\cite{rust-book}}
    \end{figure}

    \vspace*{-2mm}
    This action of creating a reference is called borrowing. As in real life, if a person owns something, you can borrow it from them. When you’re done, you have to give it back. You don’t own it.\cite{rust-book}
\end{frame}

\section{The Slice Type}
\subsection{String Slices}
\begin{frame}{String Slices: A Reference to A Part of A String}
    \begin{figure}[htpb]
        \centering
        \includegraphics[width=0.5\textwidth]{./images/slices.png}
        \vspace*{-2mm}
        \caption{Using Slices\cite{rust-book}}
    \end{figure}

    \vspace*{-5mm}
    \begin{figure}[htpb]
        \centering
        \includegraphics[width=0.5\textwidth]{./images/slices2.png}
        \vspace*{-2mm}
        \caption{Slices without the first number\cite{rust-book}}
    \end{figure}

    \vspace*{-5mm}
    \begin{figure}[htpb]
        \centering
        \includegraphics[width=0.5\textwidth]{./images/slices3.png}
        \vspace*{-2mm}
        \caption{Slices without the second number\cite{rust-book}}
    \end{figure}
\end{frame}

\begin{frame}{Slices In Memory}
    \begin{figure}[htpb]
        \centering
        \includegraphics[width=0.5\textwidth]{./images/slices4.png}
        \vspace*{-2mm}
        \caption{String slice referring to part of a String\cite{rust-book}}
    \end{figure}
\end{frame}

\section{Struct}
\subsection{Definition}
\begin{frame}
    \begin{figure}[htpb]
        \centering
        \includegraphics[width=0.5\textwidth]{./images/struct.png}
        \caption{A User struct definition\cite{rust-book}}
    \end{figure}

    \begin{figure}[htpb]
        \centering
        \includegraphics[width=0.8\textwidth]{./images/struct2.png}
        \caption{Creating an instance of the User struct\cite{rust-book}}
    \end{figure}
\end{frame}

\subsection{Methods}
\begin{frame}
    \begin{figure}[htpb]
        \centering
        \includegraphics[width=0.75\textwidth]{./images/struct3.png}
        \caption{Defining an area method on the Rectangle struct\cite{rust-book}}
    \end{figure}
\end{frame}

\section{Enum}
\begin{frame}
    \begin{figure}[htpb]
        \centering
        \includegraphics[width=\textwidth]{./images/enum.png}
        \caption{an enum and declaring two variables with it\cite{rust-book}}
    \end{figure}
\end{frame}

\subsection{Pattern Matching}
\begin{frame}{The match Control Flow Construct}
    \begin{figure}[htpb]
        \centering
        \includegraphics[width=0.7\textwidth]{./images/enum2.png}
        \caption{An enum and a match expression that has the variants of the enum as its patterns\cite{rust-book}}
    \end{figure}
\end{frame}

\section{I/O Operations}
\subsection{Reading A File}
\begin{frame}{How to Read A File}
    \begin{figure}[htpb]
        \centering
        \includegraphics[width=\textwidth]{./images/file.png}
        \caption{Reading the contents of the file specified by the second argument\cite{rust-book}}
    \end{figure}
\end{frame}

\subsection{Command-Line Arguments}
\begin{frame}{Collecting Command-Line Arguments Into A Vector}
    \begin{figure}[htpb]
        \centering
        \includegraphics[width=\textwidth]{./images/cmd.png}
        \caption{Creating variables to hold the query argument and file path argument\cite{rust-book}}
    \end{figure}
\end{frame}

\section{Common Collection}
\subsection{Vectors}
\begin{frame}
    \begin{figure}[htpb]
        \centering
        \includegraphics[width=0.8\textwidth]{./images/vec1.png}
        \caption{Creating a new, empty vector to hold values of type i32\cite{rust-book}}
    \end{figure}

    \begin{figure}[htpb]
        \centering
        \includegraphics[width=0.8\textwidth]{./images/vec2.png}
        \caption{Creating a new vector containing values\cite{rust-book}}
    \end{figure}

    \begin{figure}[htpb]
        \centering
        \includegraphics[width=0.8\textwidth]{./images/vec3.png}
        \caption{Printing each element in a vector by iterating over the elements using a for loop\cite{rust-book}}
    \end{figure}
\end{frame}

\section*{References}
\bibliography{main}
\bibliographystyle{ieeetr}
\end{document}
