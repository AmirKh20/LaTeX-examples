\documentclass[10pt]{beamer}

\usetheme{AnnArbor}
\usecolortheme{dracula}


\title{6LoWPANs: An Overview}
\author{Amir~Khayatzadeh}
\institute[IAUM]{Department of Computer Engineering\\ Islamic Azad University, Mashhad Branch}
\date{Spring 2024}

\usepackage{tikz}
\titlegraphic{
    \begin{tikzpicture}[overlay, remember picture]
    \node[right=0.05cm] at (current page.149){
        \includegraphics[scale=.1]{../images/azad-logo.png}
    };
    \node[right=1.5cm] at (current page.151){
        {\scriptsize Islamic Azad University, Mashhad Branch}
    };
    \end{tikzpicture}
}

\AtBeginSection[]
{
  \begin{frame}
    \frametitle{Table of Contents}
    \tableofcontents[currentsection]
  \end{frame}
}

\begin{document}

\begin{frame}[plain]
    \titlepage
\end{frame}


\section{Introduction}
\subsection{LoWPANs}
\begin{frame}{LLNs\cite{rfc6568}}{What Are They?}

    Low-power and lossy networks (LLNs) are networks made of highly constrained nodes:
    \begin{itemize}
        \item Limited CPU
        \item Limited Memory
        \item Limited Power
    \end{itemize}
    interconnected by a variety of "lossy" links:
    \begin{itemize}
        \item Low-power Radio Links
        \item Power-line Communication (PLC)
    \end{itemize}
    Some Characteristics:
    \begin{itemize}
        \item Low Speed
        \item Low Performance
        \item Low Cost
        \item Unstable Connectivity
    \end{itemize}

\end{frame}

\begin{frame}{LoWPANs and IEEE 802.15.4}{An Instance of LLNs}
    IEEE 802.15.4\cite{9144691} defines standards for low-rate wireless networks for the physical layer and MAC sublayer.

    Characteristics of a \underline{Lo}w-power \underline{W}ireless \underline{P}ersonal \underline{A}rea \underline{N}etwork (LoWPAN)\cite{rfc6568}:
    \begin{itemize}
        \item Limited Processing Capability
            \begin{itemize}
                \item 8, 16, 32-bit Processors
                \item Clock Rate in Order of Tens of MHz
            \end{itemize}
        \item Small Memory Capacity
            \begin{itemize}
                \item Only a Few Kilobytes of RAM
                \item A Few Dozen of Kilobytes or Megabytes of ROM
            \end{itemize}
    \end{itemize}
    \vspace*{-8mm}
    \begin{columns}
        \begin{column}{0.5\textwidth}
            \begin{figure}[htpb]
                \centering
                \includegraphics[width=0.4\textwidth]{../images/Imote2-cmt.png}
                \caption{IMote2: wireless sensor network node\cite{imote2cmt} with 256kB RAM, 32MB Flash}
            \end{figure}
        \end{column}
        \begin{column}{0.5\textwidth}
            \begin{figure}[htpb]
                \centering
                \includegraphics[width=0.4\textwidth]{../images/cc2538.png}
                \caption{CC2538: wireless microcontroller System-on-Chip (SoC) with 32kB RAM, 512kB Flash\cite{cc2538}}
            \end{figure}
        \end{column}
    \end{columns}
\end{frame}

\begin{frame}{Characteristics of LoWPANs\cite{rfc6568}}
    \begin{itemize}
        \item Low Power
            \begin{itemize}
                \item Radio Frequency Transceivers' Currents Between 10 to 30 mA
            \end{itemize}
        \item Short Range
            \begin{itemize}
                \item 10 Meters defined by IEEE 802.15.4
                \item Tens of Meters
                \item Over 100 Meters in line-of-sight situations
            \end{itemize}
        \item Low Bit Rate
            \begin{itemize}
                \item 250 kbit/s defined by IEEE 802.15.4 in the 2.4-GHz band
                \item 20, 40, 100 kbit/s
            \end{itemize}
    \end{itemize}
\end{frame}

\subsection{6LoWPANs}
\begin{frame}{About 6LoWPAN}{The IETF}
    The Internet Engineering Task Force (IETF), which is responsible for the technical standards of many Internet protocols such as HTTP, TCP, UDP, etc., has defined IPv\underline{6} over \underline{Lo}w-power \underline{W}ireless \underline{P}ersonal \underline{A}rea \underline{N}etwork (6LoWPAN) in RFC 4919\cite{rfc4919}.

    \begin{figure}[htpb]
        \centering
        \includegraphics[width=0.6\textwidth]{../images/ietf-logo.png}
        \caption{The IETF Logo\cite{IETF}}
    \end{figure}

\end{frame}

\begin{frame}{About 6LoWPAN}{Why?}
    The application of IP technology is assumed to provide the following benefits\cite{rfc4919}:
    \begin{enumerate}
        \item Existing Infrastructure
        \item Already Working Technology
        \item Open and Freely Available
        \item Tools for Diagnostics, Management, and Commissioning
        \item Connectable to Other IP-based Networks
    \end{enumerate}

    \begin{figure}[htpb]
        \centering
        \includegraphics[width=0.5\textwidth]{../images/lowpan-logo.png}
        \caption{6LoWPAN Logo}
    \end{figure}
\end{frame}

\begin{frame}{Use Cases}{The Internet of Things}
    6LoWPANs have changed IoT radically. Before, a complex application layer gateway was needed to make devices such as ZigBee, Bluetooth and proprietary systems connect to the Internet.\cite{olsson20146lowpan}

    \begin{columns}
        \begin{column}{0.5\textwidth}
            \begin{figure}[htpb]
                \centering
                \includegraphics[width=0.8\textwidth]{../images/lifx-light-bulb.jpg}
                \caption{LIFX LED Bulbs\cite{LIFX}}
            \end{figure}
        \end{column}
        \begin{column}{0.5\textwidth}
            \begin{figure}[htpb]
                \centering
                \includegraphics[width=0.8\textwidth]{../images/Tado-Thermostat.jpg}
                \caption{Tado Smart Thermostat\cite{tado-Smart-Thermostat}}
            \end{figure}
        \end{column}
    \end{columns}
\end{frame}

\begin{frame}{Network Architecture}
    \begin{figure}[htpb]
        \centering
        \includegraphics[width=0.68\textwidth]{../images/6lowpan-network-architecture-inverted.png}
        \caption{An example of an IPv6 network with a 6LoWPAN mesh network\cite{olsson20146lowpan}}
    \end{figure}
\end{frame}

\section{Challenges}
\begin{frame}
    A number of challenges are faced for applying IP on LoWPANs\cite{rfc4919}:
    \begin{enumerate}
        \item IP Connectivity
            \begin{itemize}
                \item Auto Configuration
                \item Large Address Space
                    \begin{itemize}
                        \item IPv6 has $3.4*10^{38} $ unique addresses
                    \end{itemize}
                \item Limited Packet Size
                \item Simple Interconnectivity to Other IP Networks like the Internet
            \end{itemize}
        \item Topologies
            \begin{itemize}
                \item Mesh and Star Topologies
                \item Requirements for The Routing Protocol:
                    \begin{itemize}
                        \item Low (or no) Overhead On Data Packets
                        \item Low Chattiness
                        \item Computation And Memory Requirements
                        \item Appropriate Routing In The Presence of Sleeping Nodes
                    \end{itemize}
            \end{itemize}
        \item Service Discovery
        \item Security
    \end{enumerate}
\end{frame}

\subsection{Packet Size}
\begin{frame}{Maximum Transmission Unit (MTU)}
    The MTU size for IPv6 packets over IEEE 802.15.4 is 1280 octets. However, a full IPv6 packet does not fit in an IEEE 802.15.4 frame\cite{rfc4919} because the maximum packet size in IEEE 802.15.4 is 127 bytes.

    \bigskip
    Some solutions have been applied in 6LoWPANs adaptation layer:
    \begin{itemize}
        \item Data Fragmentation and Reassembly
        \item Header Compression
    \end{itemize}
    \begin{figure}[htpb]
        \centering
        \includegraphics[width=0.8\textwidth]{../images/6lowpan-stack-inverted.png}
        \caption{The OSI model, a Wi-Fi stack example and the 6LoWPAN stack\cite{olsson20146lowpan}}
    \end{figure}

\end{frame}

\section{Routing}
\subsection{Two Categories}
\begin{frame}{Route-over (layer three) Forwarding}
    \begin{figure}[htpb]
        \centering
        \includegraphics[width=\textwidth]{../images/route-over-inverted.png}
        \caption{Route-over Forwarding in The Network Layer\cite{olsson20146lowpan}}
    \end{figure}

    \begin{itemize}
        \item Each hop represent an IP router.\cite{olsson20146lowpan}
        \item The most widely used routing protocol for route-over 6LoWPAN networks today is routing protocol for low-power and lossy networks (RPL) as defined by IETF in RFC 6550.\cite{olsson20146lowpan}
    \end{itemize}
\end{frame}

\begin{frame}{Mesh-under (layer two) Forwarding}
    \begin{figure}[htpb]
        \centering
        \includegraphics[width=\textwidth]{../images/mesh-under-inverted.png}
        \caption{Mesh-under Forwarding in The Data Link Layer\cite{olsson20146lowpan}}
    \end{figure}

    \begin{itemize}
        \item Mesh-under networks are considered to be one IP subnet.\cite{olsson20146lowpan}
        \item The only IP router in such a system is the edge router.\cite{olsson20146lowpan}
        \item Mesh-under networks are best suited for smaller and local networks.\cite{olsson20146lowpan}
    \end{itemize}
\end{frame}

\section{Design Space}
\begin{frame}
    The possible dimensions for scenario categorization\cite{rfc6568}:
    \begin{itemize}
        \item Deployment
        \item Network Size
        \item Power Source
        \item Connectivity
        \item Multi-Hop Communication
            \begin{itemize}
                \item The number of hops needed to reach the destination or the edge of the network.
                \item A single hop for simple star topology.
                \item Multi-hop communication for more elaborate topologies: meshes or trees.
            \end{itemize}
        \item Traffic Pattern
            \begin{itemize}
                \item Point-to-Multipoint (P2MP)
                \item Multipoint-to-Point (MP2P)
                \item Point-to-Point (P2P)
            \end{itemize}
        \item Security Level
        \item Mobility
        \item Quality of Service(QoS)
            \begin{itemize}
                \item Parameters for QoS should consider collective data for latency, packet loss, data throughput, etc.
            \end{itemize}
    \end{itemize}
\end{frame}

\section{Application Scenarios}
\subsection{Healthcare}
\begin{frame}{A Use Case and Its Requirements}{Example: Healthcare at Home by Tele-Assistance}
    A senior citizen who lives alone wears one to several wearable LoWPAN nodes to measure heartbeat, pulse rate, etc.\cite{rfc6568}
    \begin{figure}[htpb]
        \centering
        \includegraphics[width=0.6\textwidth]{../images/dangers-of-seniors-living alone2.jpg}
        \caption{Dangers of Seniors Living Alone.\cite{Jonahtjw_2019}}
    \end{figure}

\end{frame}

\begin{frame}{Proposed Method and Things to Consider\cite{rfc6568}}
    Proposed Method:
    \begin{itemize}
        \item Densely installed nodes for movement detection
        \item A LoWPAN border router (LBR) at home for sending information
        \item LCDs to check the data at home
    \end{itemize}

    Things to Consider:
    \begin{itemize}
        \item Node Management
            \begin{itemize}
                \item Different Duty Cycles
            \end{itemize}
        \item Multipath Interference
            \begin{itemize}
                \item Walls and Obstacles
                \item Change of Body Position During Sleep
            \end{itemize}
        \item Data Gathering.
            \begin{itemize}
                \item Periodic
                \item Event-driven: Very Time-critical
            \end{itemize}
        \item Privacy
            \begin{itemize}
                \item Secret Keys Between Sensor Nodes
                \item Role-based Access Control
            \end{itemize}
    \end{itemize}
\end{frame}

\begin{frame}{Parameters}
    Dominant parameters in healthcare applications\cite{rfc6568}:
    \begin{enumerate}
        \item Deployment: Pre-planned.
        \item Network Size: Small, high node density.
        \item Power Source: Hybrid.
        \item Connectivity: Always on.
        \item Multi-Hop Communication: Multi-hop for home-care devices; patient’s body network is star topology. Multipath interference due to walls and obstacles at home must be considered.
        \item Traffic Pattern: MP2P/P2MP (data collection), P2P (local diagnostic).
        \item Security Level: Data privacy and security must be provided. Encryption is required. It is required that role-based access control be supported by a lightweight authentication mechanism.
        \item Mobility: Moderate (patient’s mobility).
        \item QoS: High level of reliability support (life-or-death implication), role-based.
    \end{enumerate}
\end{frame}

\begin{frame}{6LoWPAN Applicability}
    \begin{figure}[htpb]
        \centering
        \includegraphics[width=\textwidth]{../images/healthcare-inverted.png}
        \caption{A Mobile Healthcare Scenario\cite{rfc6568}}
    \end{figure}
    The patient’s body network can be simply configured as a star topology with a LC dealing with data aggregation and dynamic network attachment when the patient moves around at home.\cite{rfc6568}
\end{frame}

\section*{References}
\bibliography{main}
\bibliographystyle{ieeetr}
\end{document}
